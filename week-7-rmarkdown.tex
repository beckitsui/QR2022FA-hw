% Options for packages loaded elsewhere
\PassOptionsToPackage{unicode}{hyperref}
\PassOptionsToPackage{hyphens}{url}
%
\documentclass[
]{article}
\usepackage{amsmath,amssymb}
\usepackage{lmodern}
\usepackage{iftex}
\ifPDFTeX
  \usepackage[T1]{fontenc}
  \usepackage[utf8]{inputenc}
  \usepackage{textcomp} % provide euro and other symbols
\else % if luatex or xetex
  \usepackage{unicode-math}
  \defaultfontfeatures{Scale=MatchLowercase}
  \defaultfontfeatures[\rmfamily]{Ligatures=TeX,Scale=1}
\fi
% Use upquote if available, for straight quotes in verbatim environments
\IfFileExists{upquote.sty}{\usepackage{upquote}}{}
\IfFileExists{microtype.sty}{% use microtype if available
  \usepackage[]{microtype}
  \UseMicrotypeSet[protrusion]{basicmath} % disable protrusion for tt fonts
}{}
\makeatletter
\@ifundefined{KOMAClassName}{% if non-KOMA class
  \IfFileExists{parskip.sty}{%
    \usepackage{parskip}
  }{% else
    \setlength{\parindent}{0pt}
    \setlength{\parskip}{6pt plus 2pt minus 1pt}}
}{% if KOMA class
  \KOMAoptions{parskip=half}}
\makeatother
\usepackage{xcolor}
\IfFileExists{xurl.sty}{\usepackage{xurl}}{} % add URL line breaks if available
\IfFileExists{bookmark.sty}{\usepackage{bookmark}}{\usepackage{hyperref}}
\hypersetup{
  pdftitle={QR WEEK 7: ZONING CLUSTERS ANALYSIS},
  hidelinks,
  pdfcreator={LaTeX via pandoc}}
\urlstyle{same} % disable monospaced font for URLs
\usepackage[margin=1in]{geometry}
\usepackage{color}
\usepackage{fancyvrb}
\newcommand{\VerbBar}{|}
\newcommand{\VERB}{\Verb[commandchars=\\\{\}]}
\DefineVerbatimEnvironment{Highlighting}{Verbatim}{commandchars=\\\{\}}
% Add ',fontsize=\small' for more characters per line
\usepackage{framed}
\definecolor{shadecolor}{RGB}{248,248,248}
\newenvironment{Shaded}{\begin{snugshade}}{\end{snugshade}}
\newcommand{\AlertTok}[1]{\textcolor[rgb]{0.94,0.16,0.16}{#1}}
\newcommand{\AnnotationTok}[1]{\textcolor[rgb]{0.56,0.35,0.01}{\textbf{\textit{#1}}}}
\newcommand{\AttributeTok}[1]{\textcolor[rgb]{0.77,0.63,0.00}{#1}}
\newcommand{\BaseNTok}[1]{\textcolor[rgb]{0.00,0.00,0.81}{#1}}
\newcommand{\BuiltInTok}[1]{#1}
\newcommand{\CharTok}[1]{\textcolor[rgb]{0.31,0.60,0.02}{#1}}
\newcommand{\CommentTok}[1]{\textcolor[rgb]{0.56,0.35,0.01}{\textit{#1}}}
\newcommand{\CommentVarTok}[1]{\textcolor[rgb]{0.56,0.35,0.01}{\textbf{\textit{#1}}}}
\newcommand{\ConstantTok}[1]{\textcolor[rgb]{0.00,0.00,0.00}{#1}}
\newcommand{\ControlFlowTok}[1]{\textcolor[rgb]{0.13,0.29,0.53}{\textbf{#1}}}
\newcommand{\DataTypeTok}[1]{\textcolor[rgb]{0.13,0.29,0.53}{#1}}
\newcommand{\DecValTok}[1]{\textcolor[rgb]{0.00,0.00,0.81}{#1}}
\newcommand{\DocumentationTok}[1]{\textcolor[rgb]{0.56,0.35,0.01}{\textbf{\textit{#1}}}}
\newcommand{\ErrorTok}[1]{\textcolor[rgb]{0.64,0.00,0.00}{\textbf{#1}}}
\newcommand{\ExtensionTok}[1]{#1}
\newcommand{\FloatTok}[1]{\textcolor[rgb]{0.00,0.00,0.81}{#1}}
\newcommand{\FunctionTok}[1]{\textcolor[rgb]{0.00,0.00,0.00}{#1}}
\newcommand{\ImportTok}[1]{#1}
\newcommand{\InformationTok}[1]{\textcolor[rgb]{0.56,0.35,0.01}{\textbf{\textit{#1}}}}
\newcommand{\KeywordTok}[1]{\textcolor[rgb]{0.13,0.29,0.53}{\textbf{#1}}}
\newcommand{\NormalTok}[1]{#1}
\newcommand{\OperatorTok}[1]{\textcolor[rgb]{0.81,0.36,0.00}{\textbf{#1}}}
\newcommand{\OtherTok}[1]{\textcolor[rgb]{0.56,0.35,0.01}{#1}}
\newcommand{\PreprocessorTok}[1]{\textcolor[rgb]{0.56,0.35,0.01}{\textit{#1}}}
\newcommand{\RegionMarkerTok}[1]{#1}
\newcommand{\SpecialCharTok}[1]{\textcolor[rgb]{0.00,0.00,0.00}{#1}}
\newcommand{\SpecialStringTok}[1]{\textcolor[rgb]{0.31,0.60,0.02}{#1}}
\newcommand{\StringTok}[1]{\textcolor[rgb]{0.31,0.60,0.02}{#1}}
\newcommand{\VariableTok}[1]{\textcolor[rgb]{0.00,0.00,0.00}{#1}}
\newcommand{\VerbatimStringTok}[1]{\textcolor[rgb]{0.31,0.60,0.02}{#1}}
\newcommand{\WarningTok}[1]{\textcolor[rgb]{0.56,0.35,0.01}{\textbf{\textit{#1}}}}
\usepackage{graphicx}
\makeatletter
\def\maxwidth{\ifdim\Gin@nat@width>\linewidth\linewidth\else\Gin@nat@width\fi}
\def\maxheight{\ifdim\Gin@nat@height>\textheight\textheight\else\Gin@nat@height\fi}
\makeatother
% Scale images if necessary, so that they will not overflow the page
% margins by default, and it is still possible to overwrite the defaults
% using explicit options in \includegraphics[width, height, ...]{}
\setkeys{Gin}{width=\maxwidth,height=\maxheight,keepaspectratio}
% Set default figure placement to htbp
\makeatletter
\def\fps@figure{htbp}
\makeatother
\setlength{\emergencystretch}{3em} % prevent overfull lines
\providecommand{\tightlist}{%
  \setlength{\itemsep}{0pt}\setlength{\parskip}{0pt}}
\setcounter{secnumdepth}{-\maxdimen} % remove section numbering
\ifLuaTeX
  \usepackage{selnolig}  % disable illegal ligatures
\fi

\title{QR WEEK 7: ZONING CLUSTERS ANALYSIS}
\author{}
\date{\vspace{-2.5em}2022-11-01}

\begin{document}
\maketitle

\hypertarget{part-i}{%
\paragraph{Part I}\label{part-i}}

What are different ways in which two variables might be correlated
(i.e.~direct causation, codetermination, etc\ldots) ? Please give one
example for each.

ANSWER: direct causation: cookies are very sweet because it contains a
lot of sugar (sweetness of cookie vs amount of sugar added)
codetermination: more butter and flour will make cookies have higher
calorie (calorie vs amount of different ingredients' energy ) reverse
causation: sweetness of cookie decreases because of the increase amount
of flour used (sweetness of cookie vs amount of flour) hidden variables
(confounders): adding more sugar to cookie will increase the calorie and
increase the sweetness

\hypertarget{part-ii}{%
\paragraph{Part II}\label{part-ii}}

The output of the bivariate regression of Home Energy Rating System
(HERS) Index score on annual utility cost per square foot, for our
sample of 20,452 homes with HERS ratings, tells us that there is a
constant of 0.634 and a coefficient of 0.006.

2)(a) Write the equation for this bivariate regression. y = bx + k y =
0.006x + 0.634

x = Annual utility cost per square foot y = Index score

2(b) A one unit increase in the HERS index score is associated with how
much change in the annual utility costs for a 2,000 sf home? 1 = 0.006x

\begin{Shaded}
\begin{Highlighting}[]
\NormalTok{x}\OtherTok{=}\DecValTok{1}\SpecialCharTok{/}\FloatTok{0.006}
\NormalTok{x}\SpecialCharTok{*}\DecValTok{2000}
\end{Highlighting}
\end{Shaded}

\begin{verbatim}
## [1] 333333.3
\end{verbatim}

\hypertarget{part-iii}{%
\paragraph{Part III}\label{part-iii}}

\hypertarget{summary}{%
\subparagraph{SUMMARY}\label{summary}}

We will continue to work on the FEMA claims data. Here is the CSV and
codebook:

\hypertarget{setting-up-working-environment}{%
\subparagraph{Setting up working
environment}\label{setting-up-working-environment}}

\begin{verbatim}
## 
## Attaching package: 'janitor'
\end{verbatim}

\begin{verbatim}
## The following objects are masked from 'package:stats':
## 
##     chisq.test, fisher.test
\end{verbatim}

\begin{verbatim}
## -- Attaching packages --------------------------------------- tidyverse 1.3.2 --
## v ggplot2 3.3.6     v purrr   0.3.4
## v tibble  3.1.7     v dplyr   1.0.9
## v tidyr   1.2.0     v stringr 1.4.0
## v readr   2.1.2     v forcats 0.5.1
## -- Conflicts ------------------------------------------ tidyverse_conflicts() --
## x dplyr::filter() masks stats::filter()
## x dplyr::lag()    masks stats::lag()
## 
## Attaching package: 'scales'
## 
## 
## The following object is masked from 'package:purrr':
## 
##     discard
## 
## 
## The following object is masked from 'package:readr':
## 
##     col_factor
## 
## 
## Install package "strengejacke" from GitHub (`devtools::install_github("strengejacke/strengejacke")`) to load all sj-packages at once!
\end{verbatim}

Create a new table of data from the ``MASSZONG.xls''

\begin{Shaded}
\begin{Highlighting}[]
\NormalTok{femaclaim }\OtherTok{\textless{}{-}} \FunctionTok{read.csv}\NormalTok{(}\StringTok{"fema\_claims\_random.csv"}\NormalTok{)}
\end{Highlighting}
\end{Shaded}

\begin{center}\rule{0.5\linewidth}{0.5pt}\end{center}

\hypertarget{we-would-like-to-know-the-relationship-between-the-total-amount-of-building-claims-variable-amountpaidonbuildingclaim-and-the-types-of-elevation-certificate-variable-elevationcertificateindicator.}{%
\paragraph{We would like to know the relationship between the total
amount of building claims (variable: amountpaidonbuildingclaim) and the
types of Elevation Certificate (variable:
elevationcertificateindicator).}\label{we-would-like-to-know-the-relationship-between-the-total-amount-of-building-claims-variable-amountpaidonbuildingclaim-and-the-types-of-elevation-certificate-variable-elevationcertificateindicator.}}

\begin{enumerate}
\def\labelenumi{\arabic{enumi}.}
\tightlist
\item
  Create a bar plot of types of certificates (hint: In order for the
  elevation certificate variable to be read as categorical data, you
  need to convert the variable as factor).
\item
  Plot the histogram of the amount paid on building claims.
\item
  Write out the equation for the fitted model.
\item
  Plot the relationship between two variables (hint: make sure you
  exclude NAs before you plot).
\item
  Fit the model and paste the result.
\item
  In a short paragraph, describe how these two variables are associated.
  Make sure to explicitly discuss and interpret each coefficient (the
  hypothesis testing using t-stats and p-value) and R2.
\item
  If you could add variables to increase the explanatory power of the
  model, what variable would you want to add? Why?
\end{enumerate}

\begin{Shaded}
\begin{Highlighting}[]
\NormalTok{ec }\OtherTok{\textless{}{-}}
\NormalTok{  femaclaim}\SpecialCharTok{$}\NormalTok{elevationcertificateindicator }\SpecialCharTok{\%\textgreater{}\%}
  \FunctionTok{na.omit}\NormalTok{() }\SpecialCharTok{\%\textgreater{}\%}
  \FunctionTok{as.factor}\NormalTok{() }\SpecialCharTok{\%\textgreater{}\%}
  \FunctionTok{plot}\NormalTok{()}
\end{Highlighting}
\end{Shaded}

\includegraphics{week-7-rmarkdown_files/figure-latex/unnamed-chunk-4-1.pdf}
2. Plot the histogram of the amount paid on building claims.

\begin{Shaded}
\begin{Highlighting}[]
\NormalTok{bc }\OtherTok{\textless{}{-}}\NormalTok{ (}\FunctionTok{na.omit}\NormalTok{(femaclaim}\SpecialCharTok{$}\NormalTok{amountpaidonbuildingclaim))}

\NormalTok{femaclaim }\SpecialCharTok{\%\textgreater{}\%}
  \FunctionTok{ggplot}\NormalTok{(}\FunctionTok{aes}\NormalTok{(amountpaidonbuildingclaim, }\AttributeTok{na.rm =} \ConstantTok{TRUE}\NormalTok{))}\SpecialCharTok{+}
  \FunctionTok{geom\_histogram}\NormalTok{(}\AttributeTok{bins =} \DecValTok{100}\NormalTok{)}\SpecialCharTok{+}
  \FunctionTok{xlim}\NormalTok{(}\DecValTok{0}\NormalTok{, }\DecValTok{90000}\NormalTok{)}
\end{Highlighting}
\end{Shaded}

\begin{verbatim}
## Warning: Removed 42383 rows containing non-finite values (stat_bin).
\end{verbatim}

\begin{verbatim}
## Warning: Removed 2 rows containing missing values (geom_bar).
\end{verbatim}

\includegraphics{week-7-rmarkdown_files/figure-latex/unnamed-chunk-5-1.pdf}
3. Write out the equation for the fitted model. buildingclaim =
coefficient*elevationcertificateindicator + constant y =
31,322.1-9314.2(x\_2 )+8661.3(x\_3 )+ 6015.5(x\_4) + ε

\begin{enumerate}
\def\labelenumi{\arabic{enumi}.}
\setcounter{enumi}{3}
\tightlist
\item
  Plot the relationship between two variables (hint: make sure you
  exclude NAs before you plot).
\end{enumerate}

\begin{Shaded}
\begin{Highlighting}[]
\NormalTok{cor }\OtherTok{\textless{}{-}}\NormalTok{ femaclaim }\SpecialCharTok{\%\textgreater{}\%}
  \FunctionTok{ggplot}\NormalTok{ (}\FunctionTok{aes}\NormalTok{(}\AttributeTok{x=}\NormalTok{elevationcertificateindicator, }\AttributeTok{y=}\NormalTok{amountpaidonbuildingclaim)) }\SpecialCharTok{+} \FunctionTok{ylim}\NormalTok{(}\DecValTok{1}\NormalTok{,}\DecValTok{1050000}\NormalTok{) }\SpecialCharTok{+} \FunctionTok{geom\_point}\NormalTok{(}\AttributeTok{color =} \StringTok{"blue"}\NormalTok{, }\AttributeTok{size=}\DecValTok{1}\NormalTok{, }\AttributeTok{alpha =} \FloatTok{0.05}\NormalTok{) }

\NormalTok{cor}
\end{Highlighting}
\end{Shaded}

\begin{verbatim}
## Warning: Removed 106575 rows containing missing values (geom_point).
\end{verbatim}

\includegraphics{week-7-rmarkdown_files/figure-latex/unnamed-chunk-6-1.pdf}

\begin{enumerate}
\def\labelenumi{\arabic{enumi}.}
\setcounter{enumi}{4}
\tightlist
\item
  Fit the model and paste the result.
\end{enumerate}

\begin{Shaded}
\begin{Highlighting}[]
\NormalTok{lm }\OtherTok{\textless{}{-}} \FunctionTok{lm}\NormalTok{(amountpaidonbuildingclaim}\SpecialCharTok{\textasciitilde{}}\FunctionTok{as.factor}\NormalTok{(elevationcertificateindicator), }\AttributeTok{data =}\NormalTok{ femaclaim)}
\FunctionTok{summary}\NormalTok{(lm)}
\end{Highlighting}
\end{Shaded}

\begin{verbatim}
## 
## Call:
## lm(formula = amountpaidonbuildingclaim ~ as.factor(elevationcertificateindicator), 
##     data = femaclaim)
## 
## Residuals:
##    Min     1Q Median     3Q    Max 
## -39983 -28888 -18363  10397 972957 
## 
## Coefficients:
##                                           Estimate Std. Error t value Pr(>|t|)
## (Intercept)                                31322.1      473.8  66.110  < 2e-16
## as.factor(elevationcertificateindicator)2  -9314.2     1115.6  -8.349  < 2e-16
## as.factor(elevationcertificateindicator)3   8661.3      750.7  11.538  < 2e-16
## as.factor(elevationcertificateindicator)4   6015.5     2331.5   2.580  0.00989
##                                              
## (Intercept)                               ***
## as.factor(elevationcertificateindicator)2 ***
## as.factor(elevationcertificateindicator)3 ***
## as.factor(elevationcertificateindicator)4 ** 
## ---
## Signif. codes:  0 '***' 0.001 '**' 0.01 '*' 0.05 '.' 0.1 ' ' 1
## 
## Residual standard error: 50690 on 22032 degrees of freedom
##   (106483 observations deleted due to missingness)
## Multiple R-squared:  0.01243,    Adjusted R-squared:  0.01229 
## F-statistic: 92.42 on 3 and 22032 DF,  p-value: < 2.2e-16
\end{verbatim}

\begin{enumerate}
\def\labelenumi{\arabic{enumi}.}
\setcounter{enumi}{5}
\tightlist
\item
  In a short paragraph, describe how these two variables are associated.
  Make sure to explicitly discuss and interpret each coefficient (the
  hypothesis testing using t-stats and p-value) and R2.
\end{enumerate}

-if the certificate indicator increase by 1, the amount paid on building
claim increase by 3757.2. The baseline one expects to get (with code 1)
for building claim is 3757.2. Newer certificate without BFE will give
one more building claim. -p value and t value are smaller than 0.01
significance level, thus we can reject the null hypothesis. The amount
paid on building claim and elevation certificate indicator are
correlated

\begin{enumerate}
\def\labelenumi{\arabic{enumi}.}
\setcounter{enumi}{6}
\tightlist
\item
  If you could add variables to increase the explanatory power of the
  model, what variable would you want to add? Why? Since we are looking
  at elevation certificate's relationship with building claim, I would
  also water to add elevation difference in this model
\end{enumerate}

Because the p values are larger than the critical value, we can not
reject the null hypothesis. The average total insurance amount paid in
dollars on the building has not changed over time.

\begin{center}\rule{0.5\linewidth}{0.5pt}\end{center}

\hypertarget{now-we-want-to-see-the-relationship-between-the-total-amount-of-contents-claims-variable-amountpaidoncontentsclaim-and-the-elevation-of-the-lowest-floor-variable-lowestfloorelevation.}{%
\paragraph{Now we want to see the relationship between the total amount
of contents claims (variable: amountpaidoncontentsclaim) and the
elevation of the lowest floor (variable:
lowestfloorelevation).}\label{now-we-want-to-see-the-relationship-between-the-total-amount-of-contents-claims-variable-amountpaidoncontentsclaim-and-the-elevation-of-the-lowest-floor-variable-lowestfloorelevation.}}

Plot the histogram of the total amount of insurance claims for contents.

\begin{Shaded}
\begin{Highlighting}[]
\FunctionTok{ggplot}\NormalTok{(femaclaim, }\FunctionTok{aes}\NormalTok{(}\AttributeTok{x =}\NormalTok{ amountpaidoncontentsclaim, }\AttributeTok{na.rm =} \ConstantTok{TRUE}\NormalTok{)) }\SpecialCharTok{+} \FunctionTok{geom\_histogram}\NormalTok{(}\AttributeTok{bins=}\DecValTok{100}\NormalTok{) }\SpecialCharTok{+} \FunctionTok{xlim}\NormalTok{(}\DecValTok{0}\NormalTok{,}\DecValTok{100000}\NormalTok{)}
\end{Highlighting}
\end{Shaded}

\begin{verbatim}
## Warning: Removed 79726 rows containing non-finite values (stat_bin).
\end{verbatim}

\begin{verbatim}
## Warning: Removed 2 rows containing missing values (geom_bar).
\end{verbatim}

\includegraphics{week-7-rmarkdown_files/figure-latex/unnamed-chunk-8-1.pdf}

Plot the histogram of the elevation of the lowest floors.

\begin{Shaded}
\begin{Highlighting}[]
\CommentTok{\#whole range}
\FunctionTok{ggplot}\NormalTok{(femaclaim, }\FunctionTok{aes}\NormalTok{(}\AttributeTok{x =}\NormalTok{ lowestfloorelevation, }\AttributeTok{na.rm =} \ConstantTok{TRUE}\NormalTok{)) }\SpecialCharTok{+} \FunctionTok{geom\_histogram}\NormalTok{(}\AttributeTok{bins =} \DecValTok{100}\NormalTok{) }\SpecialCharTok{+} \FunctionTok{xlim}\NormalTok{(}\SpecialCharTok{{-}}\DecValTok{100}\NormalTok{,}\DecValTok{1000}\NormalTok{)}
\end{Highlighting}
\end{Shaded}

\begin{verbatim}
## Warning: Removed 481 rows containing non-finite values (stat_bin).
\end{verbatim}

\begin{verbatim}
## Warning: Removed 2 rows containing missing values (geom_bar).
\end{verbatim}

\includegraphics{week-7-rmarkdown_files/figure-latex/unnamed-chunk-9-1.pdf}

\begin{Shaded}
\begin{Highlighting}[]
\CommentTok{\#focused range with y limit}
\FunctionTok{ggplot}\NormalTok{(femaclaim, }\FunctionTok{aes}\NormalTok{(}\AttributeTok{x =}\NormalTok{ lowestfloorelevation, }\AttributeTok{na.rm =} \ConstantTok{TRUE}\NormalTok{)) }\SpecialCharTok{+} \FunctionTok{geom\_histogram}\NormalTok{(}\AttributeTok{bins =} \DecValTok{100}\NormalTok{) }\SpecialCharTok{+} \FunctionTok{xlim}\NormalTok{(}\SpecialCharTok{{-}}\DecValTok{100}\NormalTok{,}\DecValTok{1000}\NormalTok{)}\SpecialCharTok{+}\FunctionTok{ylim}\NormalTok{(}\DecValTok{0}\NormalTok{,}\DecValTok{300}\NormalTok{)}
\end{Highlighting}
\end{Shaded}

\begin{verbatim}
## Warning: Removed 481 rows containing non-finite values (stat_bin).
\end{verbatim}

\begin{verbatim}
## Warning: Removed 11 rows containing missing values (geom_bar).
\end{verbatim}

\includegraphics{week-7-rmarkdown_files/figure-latex/unnamed-chunk-9-2.pdf}

Write out the equation for the fitted model.

-total amount of contents claims (variable: amountpaidoncontentsclaim) =
intercept + coefficient * elevation of the lowest floor (variable:
lowestfloorelevation)

Plot the relationship between two variables.

\begin{Shaded}
\begin{Highlighting}[]
\NormalTok{femaclaim }\SpecialCharTok{\%\textgreater{}\%}
  \FunctionTok{select}\NormalTok{(amountpaidoncontentsclaim, lowestfloorelevation) }\SpecialCharTok{\%\textgreater{}\%}
  \FunctionTok{na.omit}\NormalTok{() }\SpecialCharTok{\%\textgreater{}\%}
  \FunctionTok{ggplot}\NormalTok{(}\FunctionTok{aes}\NormalTok{(}\AttributeTok{x=}\NormalTok{lowestfloorelevation, }\AttributeTok{y=}\NormalTok{amountpaidoncontentsclaim)) }\SpecialCharTok{+} \FunctionTok{geom\_point}\NormalTok{()}
\end{Highlighting}
\end{Shaded}

\includegraphics{week-7-rmarkdown_files/figure-latex/unnamed-chunk-10-1.pdf}

Fit the model and paste the result.

\begin{Shaded}
\begin{Highlighting}[]
\NormalTok{lm\_2 }\OtherTok{\textless{}{-}} \FunctionTok{lm}\NormalTok{(amountpaidoncontentsclaim}\SpecialCharTok{\textasciitilde{}}\NormalTok{lowestfloorelevation, }\AttributeTok{data =}\NormalTok{ femaclaim)}
\FunctionTok{summary}\NormalTok{(lm\_2)}
\end{Highlighting}
\end{Shaded}

\begin{verbatim}
## 
## Call:
## lm(formula = amountpaidoncontentsclaim ~ lowestfloorelevation, 
##     data = femaclaim)
## 
## Residuals:
##    Min     1Q Median     3Q    Max 
## -25827 -12837  -9120   1416 485880 
## 
## Coefficients:
##                       Estimate Std. Error t value Pr(>|t|)    
## (Intercept)          1.412e+04  1.300e+02 108.575  < 2e-16 ***
## lowestfloorelevation 1.171e+00  3.377e-01   3.468 0.000526 ***
## ---
## Signif. codes:  0 '***' 0.001 '**' 0.01 '*' 0.05 '.' 0.1 ' ' 1
## 
## Residual standard error: 28760 on 49130 degrees of freedom
##   (79387 observations deleted due to missingness)
## Multiple R-squared:  0.0002447,  Adjusted R-squared:  0.0002243 
## F-statistic: 12.02 on 1 and 49130 DF,  p-value: 0.0005258
\end{verbatim}

In a short paragraph, describe how these two variables are associated.
Make sure to explicitly discuss and interpret each coefficient (the
hypothesis testing using t-stats and p-value) and R2.

-The intercept is 1412, indicating that the baseline (min number of
lowest floor level) gets 1412 total amount of contents claims. With one
level of floor elevation change, the amount of claim increase by 1.17.
The p-value and t value are below significance level 0.01, hence we can
reject the null hypothesis. The total amount of contents claims and
elevation of the lowest floor are correlated.

If you could add variables to increase the explanatory power of the
model, what variable would you want to add? Why?

-I would include the number of floors in the insured building, because
this model concerns with the floor levels' relationship with amount of
claim for contents

\end{document}
